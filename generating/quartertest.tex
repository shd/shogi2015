\documentclass[12pt]{scrartcl}
\usepackage[russian]{babel}
\usepackage{CJKutf8}
\usepackage[usebaselinestretch]{CJKvert}
\usepackage[utf8]{inputenc}
\usepackage[width=18cm, height=28cm, top=0.5cm,bottom=0.5cm]{geometry}
\usepackage{rotating}
\usepackage[T2A]{fontenc}
\usepackage{sho}
\usepackage{setspace}
\usepackage{color}

\definecolor{light-gray}{gray}{0.5}
\definecolor{name}{gray}{0.5}

\pagenumbering{gobble}% Remove page numbers (and reset to 1)

\newcommand\testwork[8]{
%\hspace{1cm}
\vspace{1cm}
{\Large 五級将棋クラブ}
\hfill\vspace{-6mm}
{\bfseries\large {#1}}
{\begin{center}\Huge 学期末試験\end{center}}
\vspace{-2mm}
{\begin{center}{\large 平成二十七年十月二十九日}\\
\end{center}}
%\vspace{-2mm}
\vspace{5mm}


\begin{turn}{90}\begin{minipage}{20cm}
\begin{center}
{\setstretch{1.2}
\vspace{-1cm}
%\large Кружок сёги ЛНМО\\
{\bfseries\large\scshape Контрольная работа за первую четверть}\\
{29 октября 2015 г.}\\
{\itshape\tiny #2}}
\end{center}

\vspace{0.5cm}

\setstretch{1}

\begin{tabular}{lp{2cm}lp{2cm}}
\begin{minipage}[t]{8cm}{\bfseries Задание 1}

\vspace{0.5cm}
Дано описание позиции:

{\it Белые:} #3

{\it Чёрные:} #4
\vspace{5mm}

Заполните диаграмму в соответствии с данным описанием.
\vspace{-3mm}
\begin{center}\shodiag{1}{}\end{center}

Используйте латинские
обозначения для фигур. Фигуры противника обводите в кружок.
%\vspace{1cm}

\end{minipage}

& &

\begin{minipage}[t]{8cm}{\bfseries Задание 2}

\vspace{0.5cm}
Опишите позицию: перечислите все фигуры, нарисованные на диаграмме, 
с указанием их положения.

\vspace{-3mm}
\begin{center}\shodiag{1}{#5}\end{center}

\vspace{5mm}

{\it Белые: }
\vspace{1cm}

{\it Чёрные: }

%\vspace{1cm}



\end{minipage} 

\end{tabular}

\vspace{0.3cm}

\begin{center}{\scshape Кружок сёги ЛНМО, 5 класс}\end{center}

\end{minipage}


\end{turn}


\vspace{1cm}

{\Huge \begin{center}\color{light-gray}数学連続教育研究所\end{center}}
\clearpage
\vspace{-1cm}
{\large 学期末試験 \hfill 第二頁}\\
\begin{turn}{90}\begin{minipage}{27.5cm}

\begin{center}
{\setstretch{1.2}
%\vspace{cm}
%\large Кружок сёги ЛНМО\\
{\large\scshape Контрольная работа за первую четверть}\\
%{29 октября 2015 г.}\\
%{\itshape\tiny #2}
}
\end{center}

\setstretch{1.1}

\vspace{2cm}

\begin{tabular}{lp{0.2cm}lp{0.2cm}l}
\begin{minipage}[t]{8cm}{\bfseries Задание 3}

%\vspace{5mm}

\begin{center}\shodiag{1}{#6}\end{center}

Решите цумэ-сёги (мат в 1 ход). Ход чёрных:
\end{minipage} & & %
\begin{minipage}[t]{8cm}{\bfseries Задание 4}
\begin{center}\shodiag{1}{#7}\end{center}

Решите цумэ-сёги (мат в 1 ход). Ход чёрных:
\end{minipage} & & %
\begin{minipage}[t]{8cm}{\bfseries Задание 5}
\begin{center}\shodiag{1}{#8}\end{center}

Решите цумэ-сёги (мат в 3 хода). Ход чёрных, наилучший ответ белых, ход чёрных:
\end{minipage} 

\end{tabular}

\vspace{3cm}
\begin{center}{\scshape Страница 2}\end{center}

\end{minipage}
\end{turn}

\clearpage
}

\begin{document}

\begin{CJK}{UTF8}{min}


\CJKvert

\testwork{エヴドキモヴァ・アリョナ}{Eвдокимова Алёна}{K42, R39, L45, +L76}{K11, +R65, +P98, P16}{\shopiece{2}{8}{\Ks}\shopiece{3}{3}{\Bs}\shopiece{6}{9}{\Ss}\shopiece{4}{8}{\Ps}\shopiece{8}{9}{\Kg}\shopiece{9}{4}{\Bpg}\shopiece{8}{5}{\Npg}\shopiece{5}{9}{\Spg}}{\shopiece{4}{4}{\Gs}\shopiece{3}{7}{\Bs}\shopiece{1}{2}{\Rs}\shopiece{2}{3}{\Pg}\shopiece{2}{4}{\Kg}}{\shopiece{2}{4}{\Bps}\shopiece{2}{6}{\Bps}\shopiece{-1}{9}{\Gs}\shopiece{4}{3}{\Kg}\shopiece{5}{4}{\Pg}}{\shopiece{3}{2}{\Rps}\shopiece{1}{5}{\Gs}\shopiece{-1}{9}{\Ns}\shopiece{1}{1}{\Lg}\shopiece{1}{3}{\Kg}\shopiece{2}{2}{\Lg}\shopiece{2}{3}{\Pg}\shopiece{4}{6}{\Bpg}}
\testwork{カラコゾヴ・パヴェル}{Каракозов Павел}{K65, +R17, B72, L87}{K77, R24, L14, +P16}{\shopiece{1}{9}{\Ks}\shopiece{2}{5}{\Rps}\shopiece{5}{3}{\Ps}\shopiece{5}{1}{\Ss}\shopiece{3}{4}{\Kg}\shopiece{4}{8}{\Bpg}\shopiece{2}{2}{\Pg}\shopiece{9}{2}{\Lpg}}{\shopiece{4}{4}{\Gs}\shopiece{3}{7}{\Bs}\shopiece{1}{2}{\Rs}\shopiece{2}{3}{\Pg}\shopiece{2}{4}{\Kg}}{\shopiece{2}{4}{\Bps}\shopiece{2}{6}{\Bps}\shopiece{-1}{9}{\Gs}\shopiece{4}{3}{\Kg}\shopiece{3}{2}{\Ng}}{\shopiece{3}{2}{\Rps}\shopiece{1}{5}{\Gs}\shopiece{-1}{9}{\Ns}\shopiece{1}{1}{\Lg}\shopiece{1}{3}{\Kg}\shopiece{2}{2}{\Lg}\shopiece{2}{3}{\Pg}\shopiece{4}{6}{\Bpg}}
\testwork{スホムリン・フィオドル}{Сухомлин Фёдор}{K45, R76, +B71, +L15}{K27, B78, P47, +L42}{\shopiece{6}{8}{\Ks}\shopiece{6}{3}{\Bs}\shopiece{1}{6}{\Nps}\shopiece{4}{2}{\Ps}\shopiece{2}{7}{\Kg}\shopiece{8}{9}{\Bg}\shopiece{8}{6}{\Sg}\shopiece{6}{2}{\Gg}}{\shopiece{1}{2}{\Rps}\shopiece{4}{5}{\Gs}\shopiece{3}{7}{\Bs}\shopiece{2}{4}{\Kg}\shopiece{2}{5}{\Pg}}{\shopiece{2}{4}{\Bps}\shopiece{2}{6}{\Bps}\shopiece{-1}{9}{\Gs}\shopiece{4}{3}{\Kg}\shopiece{3}{2}{\Ng}}{\shopiece{3}{2}{\Rps}\shopiece{1}{5}{\Gs}\shopiece{-1}{9}{\Ns}\shopiece{1}{1}{\Lg}\shopiece{1}{3}{\Kg}\shopiece{2}{2}{\Lg}\shopiece{2}{3}{\Pg}\shopiece{4}{6}{\Bpg}}
\testwork{チストヴ・オレグ}{Чистов Олег}{K91, +B94, S12, P56}{K36, R28, G85, +P76}{\shopiece{7}{1}{\Ks}\shopiece{6}{6}{\Rps}\shopiece{2}{5}{\Gs}\shopiece{1}{1}{\Bps}\shopiece{4}{9}{\Kg}\shopiece{3}{7}{\Bg}\shopiece{6}{1}{\Ppg}\shopiece{5}{8}{\Pg}}{\shopiece{1}{3}{\Rs}\shopiece{4}{5}{\Gs}\shopiece{3}{7}{\Bs}\shopiece{2}{4}{\Kg}\shopiece{2}{5}{\Pg}}{\shopiece{2}{4}{\Bps}\shopiece{2}{6}{\Bps}\shopiece{-1}{9}{\Gs}\shopiece{4}{3}{\Kg}\shopiece{5}{2}{\Pg}\shopiece{3}{2}{\Pg}}{\shopiece{3}{2}{\Rps}\shopiece{2}{5}{\Gs}\shopiece{-1}{9}{\Ps}\shopiece{-1}{8}{\Ns}\shopiece{1}{2}{\Lg}\shopiece{1}{3}{\Kg}\shopiece{2}{3}{\Pg}\shopiece{2}{4}{\Ng}}
\testwork{ベクレネヴァ・ダリア}{Бекренева Дарья}{K42, +R52, G47, +P12}{K92, +R69, +B84, +P63}{\shopiece{9}{7}{\Ks}\shopiece{5}{4}{\Bps}\shopiece{9}{3}{\Pps}\shopiece{5}{5}{\Rs}\shopiece{1}{4}{\Kg}\shopiece{7}{5}{\Rg}\shopiece{3}{9}{\Sg}\shopiece{3}{7}{\Ppg}}{\shopiece{4}{4}{\Gs}\shopiece{3}{7}{\Bs}\shopiece{1}{2}{\Rs}\shopiece{2}{3}{\Pg}\shopiece{2}{4}{\Kg}}{\shopiece{2}{4}{\Bps}\shopiece{2}{6}{\Bps}\shopiece{-1}{9}{\Gs}\shopiece{4}{3}{\Kg}\shopiece{5}{4}{\Pg}}{\shopiece{3}{2}{\Rps}\shopiece{2}{5}{\Gs}\shopiece{2}{7}{\Ps}\shopiece{-1}{9}{\Ps}\shopiece{-1}{8}{\Ns}\shopiece{1}{2}{\Lg}\shopiece{1}{3}{\Kg}\shopiece{1}{5}{\Pg}\shopiece{2}{3}{\Pg}\shopiece{1}{7}{\Rpg}}
\testwork{ボルゾフ・アルテミイ}{Борзов Артемий}{K74, B36, +L42, +S71}{K49, R33, G89, +N96}{\shopiece{1}{6}{\Ks}\shopiece{3}{1}{\Rs}\shopiece{3}{3}{\Sps}\shopiece{6}{8}{\Ps}\shopiece{6}{3}{\Kg}\shopiece{6}{7}{\Rg}\shopiece{2}{4}{\Gg}\shopiece{7}{3}{\Npg}}{\shopiece{1}{3}{\Rs}\shopiece{4}{5}{\Gs}\shopiece{3}{7}{\Bs}\shopiece{2}{4}{\Kg}\shopiece{2}{5}{\Pg}}{\shopiece{2}{4}{\Bps}\shopiece{2}{6}{\Bps}\shopiece{-1}{9}{\Gs}\shopiece{4}{3}{\Kg}\shopiece{5}{2}{\Pg}\shopiece{3}{2}{\Pg}}{\shopiece{3}{2}{\Rps}\shopiece{2}{5}{\Gs}\shopiece{-1}{9}{\Ps}\shopiece{-1}{8}{\Ns}\shopiece{1}{2}{\Lg}\shopiece{1}{3}{\Kg}\shopiece{2}{3}{\Pg}\shopiece{2}{4}{\Ng}}
\testwork{ルビナ・オレシャ}{Рубина Олеся}{K18, R41, P39, P54}{K82, B87, L37, G43}{\shopiece{7}{9}{\Ks}\shopiece{8}{2}{\Rps}\shopiece{3}{4}{\Pps}\shopiece{8}{8}{\Pps}\shopiece{3}{8}{\Kg}\shopiece{4}{9}{\Bg}\shopiece{2}{7}{\Lg}\shopiece{9}{4}{\Ng}}{\shopiece{4}{4}{\Gs}\shopiece{3}{7}{\Bs}\shopiece{1}{2}{\Rs}\shopiece{2}{3}{\Pg}\shopiece{2}{4}{\Kg}}{\shopiece{2}{4}{\Bps}\shopiece{2}{6}{\Bps}\shopiece{-1}{9}{\Gs}\shopiece{4}{3}{\Kg}\shopiece{3}{2}{\Ng}}{\shopiece{3}{2}{\Rps}\shopiece{1}{5}{\Gs}\shopiece{-1}{9}{\Ns}\shopiece{1}{1}{\Lg}\shopiece{1}{3}{\Kg}\shopiece{2}{2}{\Lg}\shopiece{2}{3}{\Pg}\shopiece{4}{6}{\Bpg}}
\testwork{シェロヴ・ナザル}{Шеров Назар}{K27, R67, P25, +L96}{K49, +B64, +R52, G47}{\shopiece{9}{8}{\Ks}\shopiece{2}{1}{\Rs}\shopiece{1}{3}{\Sps}\shopiece{2}{4}{\Ps}\shopiece{3}{3}{\Kg}\shopiece{6}{5}{\Rg}\shopiece{1}{6}{\Gg}\shopiece{8}{7}{\Ng}}{\shopiece{1}{2}{\Rps}\shopiece{4}{5}{\Gs}\shopiece{3}{7}{\Bs}\shopiece{2}{4}{\Kg}\shopiece{2}{5}{\Pg}}{\shopiece{2}{4}{\Bps}\shopiece{2}{6}{\Bps}\shopiece{-1}{9}{\Gs}\shopiece{4}{3}{\Kg}\shopiece{5}{2}{\Pg}\shopiece{3}{2}{\Pg}}{\shopiece{3}{2}{\Rps}\shopiece{2}{5}{\Gs}\shopiece{2}{7}{\Ps}\shopiece{-1}{9}{\Ps}\shopiece{-1}{8}{\Ns}\shopiece{1}{2}{\Lg}\shopiece{1}{3}{\Kg}\shopiece{1}{5}{\Pg}\shopiece{2}{3}{\Pg}\shopiece{1}{7}{\Rpg}}
\testwork{ネマトヴァ・リナタ}{Нематова Рената}{K95, R29, +P33, +B88}{K34, R18, +N41, G42}{\shopiece{6}{9}{\Ks}\shopiece{8}{7}{\Bs}\shopiece{8}{6}{\Pps}\shopiece{5}{2}{\Ps}\shopiece{2}{6}{\Kg}\shopiece{7}{8}{\Rg}\shopiece{5}{7}{\Bpg}\shopiece{9}{3}{\Sg}}{\shopiece{1}{3}{\Rs}\shopiece{4}{5}{\Gs}\shopiece{3}{7}{\Bs}\shopiece{2}{4}{\Kg}\shopiece{2}{5}{\Pg}}{\shopiece{2}{4}{\Bps}\shopiece{2}{6}{\Bps}\shopiece{-1}{9}{\Gs}\shopiece{4}{3}{\Kg}\shopiece{5}{4}{\Pg}}{\shopiece{3}{2}{\Rps}\shopiece{1}{5}{\Gs}\shopiece{-1}{9}{\Ns}\shopiece{1}{1}{\Lg}\shopiece{1}{3}{\Kg}\shopiece{2}{2}{\Lg}\shopiece{2}{3}{\Pg}\shopiece{4}{6}{\Bpg}}
\testwork{イエメリアヌチク・ウラヂミル}{Емельянчик Владимир}{K14, +B32, G58, P24}{K82, +R23, P37, +P38}{\shopiece{2}{7}{\Ks}\shopiece{9}{3}{\Bs}\shopiece{8}{6}{\Rps}\shopiece{8}{9}{\Ss}\shopiece{5}{9}{\Kg}\shopiece{2}{4}{\Bg}\shopiece{2}{6}{\Pg}\shopiece{7}{2}{\Ppg}}{\shopiece{1}{2}{\Rps}\shopiece{4}{5}{\Gs}\shopiece{3}{7}{\Bs}\shopiece{2}{4}{\Kg}\shopiece{2}{5}{\Pg}}{\shopiece{2}{4}{\Bps}\shopiece{2}{6}{\Bps}\shopiece{-1}{9}{\Gs}\shopiece{4}{3}{\Kg}\shopiece{5}{2}{\Pg}\shopiece{3}{2}{\Pg}}{\shopiece{3}{2}{\Rps}\shopiece{1}{5}{\Gs}\shopiece{-1}{9}{\Ns}\shopiece{1}{1}{\Lg}\shopiece{1}{3}{\Kg}\shopiece{2}{2}{\Lg}\shopiece{2}{3}{\Pg}\shopiece{4}{6}{\Bpg}}
\testwork{トルカチョヴ・キリル}{Толкачёв Кирилл}{K99, +B71, +P43, G82}{K68, +R48, P93, B66}{\shopiece{2}{3}{\Ks}\shopiece{3}{4}{\Bs}\shopiece{6}{9}{\Pps}\shopiece{5}{4}{\Pps}\shopiece{1}{9}{\Kg}\shopiece{5}{8}{\Bpg}\shopiece{9}{8}{\Sg}\shopiece{5}{7}{\Ppg}}{\shopiece{1}{2}{\Rps}\shopiece{4}{5}{\Gs}\shopiece{3}{7}{\Bs}\shopiece{2}{4}{\Kg}\shopiece{2}{5}{\Pg}}{\shopiece{2}{4}{\Bps}\shopiece{2}{6}{\Bps}\shopiece{-1}{9}{\Gs}\shopiece{4}{3}{\Kg}\shopiece{5}{4}{\Pg}}{\shopiece{3}{2}{\Rps}\shopiece{2}{5}{\Gs}\shopiece{2}{7}{\Ps}\shopiece{-1}{9}{\Ps}\shopiece{-1}{8}{\Ns}\shopiece{1}{2}{\Lg}\shopiece{1}{3}{\Kg}\shopiece{1}{5}{\Pg}\shopiece{2}{3}{\Pg}\shopiece{1}{7}{\Rpg}}
\testwork{アンヅリアノヴ・イリア}{Андрианов Илья}{K31, B67, L63, G75}{K78, R53, P18, +B55}{\shopiece{3}{1}{\Ks}\shopiece{2}{4}{\Rs}\shopiece{1}{2}{\Bps}\shopiece{4}{7}{\Pps}\shopiece{5}{9}{\Kg}\shopiece{2}{1}{\Bpg}\shopiece{2}{3}{\Ppg}\shopiece{2}{5}{\Ng}}{\shopiece{1}{2}{\Rps}\shopiece{4}{5}{\Gs}\shopiece{3}{7}{\Bs}\shopiece{2}{4}{\Kg}\shopiece{2}{5}{\Pg}}{\shopiece{2}{4}{\Bps}\shopiece{2}{6}{\Bps}\shopiece{-1}{9}{\Gs}\shopiece{4}{3}{\Kg}\shopiece{3}{2}{\Ng}}{\shopiece{3}{2}{\Rps}\shopiece{1}{5}{\Gs}\shopiece{-1}{9}{\Ns}\shopiece{1}{1}{\Lg}\shopiece{1}{3}{\Kg}\shopiece{2}{2}{\Lg}\shopiece{2}{3}{\Pg}\shopiece{4}{6}{\Bpg}}
\testwork{ダニロヴァ・リュボヴィ}{Данилова Любовь}{K29, +R66, P63, +N75}{K36, +B56, +P47, G35}{\shopiece{6}{4}{\Ks}\shopiece{7}{8}{\Bs}\shopiece{8}{9}{\Rps}\shopiece{2}{4}{\Pps}\shopiece{4}{2}{\Kg}\shopiece{6}{6}{\Rpg}\shopiece{4}{7}{\Pg}\shopiece{6}{2}{\Pg}}{\shopiece{1}{2}{\Rps}\shopiece{4}{5}{\Gs}\shopiece{3}{7}{\Bs}\shopiece{2}{4}{\Kg}\shopiece{2}{5}{\Pg}}{\shopiece{2}{4}{\Bps}\shopiece{2}{6}{\Bps}\shopiece{-1}{9}{\Gs}\shopiece{4}{3}{\Kg}\shopiece{5}{2}{\Pg}\shopiece{3}{2}{\Pg}}{\shopiece{3}{2}{\Rps}\shopiece{2}{5}{\Gs}\shopiece{2}{7}{\Ps}\shopiece{-1}{9}{\Ps}\shopiece{-1}{8}{\Ns}\shopiece{1}{2}{\Lg}\shopiece{1}{3}{\Kg}\shopiece{1}{5}{\Pg}\shopiece{2}{3}{\Pg}\shopiece{1}{7}{\Rpg}}
\testwork{チェクマリョヴ・ヂミツリイ}{Чекмарёв Дмитрий}{K61, B19, G17, +P52}{K66, B73, +R85, +S39}{\shopiece{4}{6}{\Ks}\shopiece{5}{2}{\Rs}\shopiece{4}{9}{\Ps}\shopiece{6}{7}{\Ps}\shopiece{9}{2}{\Kg}\shopiece{8}{2}{\Rg}\shopiece{1}{3}{\Ppg}\shopiece{3}{6}{\Lpg}}{\shopiece{1}{2}{\Rps}\shopiece{4}{5}{\Gs}\shopiece{3}{7}{\Bs}\shopiece{2}{4}{\Kg}\shopiece{2}{5}{\Pg}}{\shopiece{2}{4}{\Bps}\shopiece{2}{6}{\Bps}\shopiece{-1}{9}{\Gs}\shopiece{4}{3}{\Kg}\shopiece{5}{2}{\Pg}\shopiece{3}{2}{\Pg}}{\shopiece{3}{2}{\Rps}\shopiece{1}{5}{\Gs}\shopiece{-1}{9}{\Ns}\shopiece{1}{1}{\Lg}\shopiece{1}{3}{\Kg}\shopiece{2}{2}{\Lg}\shopiece{2}{3}{\Pg}\shopiece{4}{6}{\Bpg}}
\testwork{パルホメンコ・ゲオルギイ}{Пархоменко Георгий}{K58, +B55, G37, P52}{K73, R34, +P32, P84}{\shopiece{5}{3}{\Ks}\shopiece{2}{4}{\Bps}\shopiece{6}{5}{\Ps}\shopiece{9}{5}{\Gs}\shopiece{1}{3}{\Kg}\shopiece{8}{4}{\Bpg}\shopiece{9}{6}{\Ppg}\shopiece{2}{2}{\Rpg}}{\shopiece{4}{4}{\Gs}\shopiece{3}{7}{\Bs}\shopiece{1}{2}{\Rs}\shopiece{2}{3}{\Pg}\shopiece{2}{4}{\Kg}}{\shopiece{2}{4}{\Bps}\shopiece{2}{6}{\Bps}\shopiece{-1}{9}{\Gs}\shopiece{4}{3}{\Kg}\shopiece{5}{2}{\Pg}\shopiece{3}{2}{\Pg}}{\shopiece{3}{2}{\Rps}\shopiece{2}{5}{\Gs}\shopiece{2}{7}{\Ps}\shopiece{-1}{9}{\Ps}\shopiece{-1}{8}{\Ns}\shopiece{1}{2}{\Lg}\shopiece{1}{3}{\Kg}\shopiece{1}{5}{\Pg}\shopiece{2}{3}{\Pg}\shopiece{1}{7}{\Rpg}}
\testwork{ヴァシェンコヴ・アレクセイ}{Ващенков Алексей}{K13, R18, N69, G68}{K75, +B54, L47, P93}{\shopiece{2}{9}{\Ks}\shopiece{4}{3}{\Rs}\shopiece{9}{7}{\Bps}\shopiece{1}{4}{\Nps}\shopiece{5}{8}{\Kg}\shopiece{2}{8}{\Bg}\shopiece{5}{2}{\Lpg}\shopiece{1}{3}{\Pg}}{\shopiece{4}{4}{\Gs}\shopiece{3}{7}{\Bs}\shopiece{1}{2}{\Rs}\shopiece{2}{3}{\Pg}\shopiece{2}{4}{\Kg}}{\shopiece{2}{4}{\Bps}\shopiece{2}{6}{\Bps}\shopiece{-1}{9}{\Gs}\shopiece{4}{3}{\Kg}\shopiece{5}{2}{\Pg}\shopiece{3}{2}{\Pg}}{\shopiece{3}{2}{\Rps}\shopiece{2}{5}{\Gs}\shopiece{-1}{9}{\Ps}\shopiece{-1}{8}{\Ns}\shopiece{1}{2}{\Lg}\shopiece{1}{3}{\Kg}\shopiece{2}{3}{\Pg}\shopiece{2}{4}{\Ng}}
\testwork{チェミナワ・ソフィア}{Чеминава София}{K33, R36, +P37, +L17}{K25, +B71, P95, +P65}{\shopiece{6}{3}{\Ks}\shopiece{6}{9}{\Rps}\shopiece{9}{8}{\Ss}\shopiece{3}{5}{\Ns}\shopiece{9}{9}{\Kg}\shopiece{5}{8}{\Bg}\shopiece{2}{3}{\Pg}\shopiece{2}{8}{\Lpg}}{\shopiece{1}{3}{\Rs}\shopiece{4}{5}{\Gs}\shopiece{3}{7}{\Bs}\shopiece{2}{4}{\Kg}\shopiece{2}{5}{\Pg}}{\shopiece{2}{4}{\Bps}\shopiece{2}{6}{\Bps}\shopiece{-1}{9}{\Gs}\shopiece{4}{3}{\Kg}\shopiece{5}{4}{\Pg}}{\shopiece{3}{2}{\Rps}\shopiece{1}{5}{\Gs}\shopiece{-1}{9}{\Ns}\shopiece{1}{1}{\Lg}\shopiece{1}{3}{\Kg}\shopiece{2}{2}{\Lg}\shopiece{2}{3}{\Pg}\shopiece{4}{6}{\Bpg}}
\testwork{ドニャマリエヴ・ルスラヌ}{Дунямалиев Руслан}{K44, +B48, +P39, +N34}{K42, B99, N65, +P57}{\shopiece{3}{5}{\Ks}\shopiece{3}{7}{\Rs}\shopiece{1}{7}{\Ps}\shopiece{3}{8}{\Lps}\shopiece{2}{9}{\Kg}\shopiece{1}{3}{\Rg}\shopiece{6}{8}{\Ppg}\shopiece{8}{8}{\Lpg}}{\shopiece{1}{2}{\Rps}\shopiece{4}{5}{\Gs}\shopiece{3}{7}{\Bs}\shopiece{2}{4}{\Kg}\shopiece{2}{5}{\Pg}}{\shopiece{2}{4}{\Bps}\shopiece{2}{6}{\Bps}\shopiece{-1}{9}{\Gs}\shopiece{4}{3}{\Kg}\shopiece{3}{2}{\Ng}}{\shopiece{3}{2}{\Rps}\shopiece{1}{5}{\Gs}\shopiece{-1}{9}{\Ns}\shopiece{1}{1}{\Lg}\shopiece{1}{3}{\Kg}\shopiece{2}{2}{\Lg}\shopiece{2}{3}{\Pg}\shopiece{4}{6}{\Bpg}}
\testwork{イワノヴ・アントヌ・オレゴヴィチ}{Иванов Антон Олегович}{K23, +R67, L83, L75}{K97, +R37, +P14, +P62}{\shopiece{8}{7}{\Ks}\shopiece{8}{2}{\Rs}\shopiece{3}{4}{\Ps}\shopiece{2}{2}{\Gs}\shopiece{2}{5}{\Kg}\shopiece{7}{9}{\Rg}\shopiece{8}{3}{\Lg}\shopiece{7}{4}{\Ppg}}{\shopiece{4}{4}{\Gs}\shopiece{3}{7}{\Bs}\shopiece{1}{2}{\Rs}\shopiece{2}{3}{\Pg}\shopiece{2}{4}{\Kg}}{\shopiece{2}{4}{\Bps}\shopiece{2}{6}{\Bps}\shopiece{-1}{9}{\Gs}\shopiece{4}{3}{\Kg}\shopiece{3}{2}{\Ng}}{\shopiece{3}{2}{\Rps}\shopiece{2}{5}{\Gs}\shopiece{2}{7}{\Ps}\shopiece{-1}{9}{\Ps}\shopiece{-1}{8}{\Ns}\shopiece{1}{2}{\Lg}\shopiece{1}{3}{\Kg}\shopiece{1}{5}{\Pg}\shopiece{2}{3}{\Pg}\shopiece{1}{7}{\Rpg}}
\testwork{セレブリャコヴ・イエヴゲニイ}{Серебряков Евгений}{K19, B63, +P13, L44}{K17, +R68, P98, B11}{\shopiece{5}{7}{\Ks}\shopiece{3}{3}{\Rps}\shopiece{2}{9}{\Bs}\shopiece{3}{9}{\Pps}\shopiece{6}{5}{\Kg}\shopiece{1}{8}{\Rg}\shopiece{7}{4}{\Ppg}\shopiece{4}{2}{\Ppg}}{\shopiece{4}{4}{\Gs}\shopiece{3}{7}{\Bs}\shopiece{1}{2}{\Rs}\shopiece{2}{3}{\Pg}\shopiece{2}{4}{\Kg}}{\shopiece{2}{4}{\Bps}\shopiece{2}{6}{\Bps}\shopiece{-1}{9}{\Gs}\shopiece{4}{3}{\Kg}\shopiece{3}{2}{\Ng}}{\shopiece{3}{2}{\Rps}\shopiece{2}{5}{\Gs}\shopiece{-1}{9}{\Ps}\shopiece{-1}{8}{\Ns}\shopiece{1}{2}{\Lg}\shopiece{1}{3}{\Kg}\shopiece{2}{3}{\Pg}\shopiece{2}{4}{\Ng}}
\testwork{ヴェショルコヴァ・ヴァルヴァラ}{Веселкова Варвара}{K43, R65, B78, +P92}{K89, +B53, +N61, P62}{\shopiece{6}{3}{\Ks}\shopiece{8}{2}{\Rps}\shopiece{2}{8}{\Ps}\shopiece{2}{1}{\Ss}\shopiece{2}{7}{\Kg}\shopiece{2}{5}{\Rpg}\shopiece{4}{7}{\Ppg}\shopiece{5}{7}{\Spg}}{\shopiece{1}{3}{\Rs}\shopiece{4}{5}{\Gs}\shopiece{3}{7}{\Bs}\shopiece{2}{4}{\Kg}\shopiece{2}{5}{\Pg}}{\shopiece{2}{4}{\Bps}\shopiece{2}{6}{\Bps}\shopiece{-1}{9}{\Gs}\shopiece{4}{3}{\Kg}\shopiece{5}{2}{\Pg}\shopiece{3}{2}{\Pg}}{\shopiece{3}{2}{\Rps}\shopiece{1}{5}{\Gs}\shopiece{-1}{9}{\Ns}\shopiece{1}{1}{\Lg}\shopiece{1}{3}{\Kg}\shopiece{2}{2}{\Lg}\shopiece{2}{3}{\Pg}\shopiece{4}{6}{\Bpg}}
\testwork{イワノヴ・アントヌ・アヌトノヴィチ}{Иванов Антон Антонович}{K41, +R34, N13, +P24}{K59, B87, P74, +S27}{\shopiece{9}{8}{\Ks}\shopiece{7}{5}{\Bps}\shopiece{4}{3}{\Pps}\shopiece{7}{7}{\Ps}\shopiece{9}{4}{\Kg}\shopiece{7}{8}{\Bpg}\shopiece{9}{3}{\Spg}\shopiece{6}{3}{\Ppg}}{\shopiece{4}{4}{\Gs}\shopiece{3}{7}{\Bs}\shopiece{1}{2}{\Rs}\shopiece{2}{3}{\Pg}\shopiece{2}{4}{\Kg}}{\shopiece{2}{4}{\Bps}\shopiece{2}{6}{\Bps}\shopiece{-1}{9}{\Gs}\shopiece{4}{3}{\Kg}\shopiece{5}{2}{\Pg}\shopiece{3}{2}{\Pg}}{\shopiece{3}{2}{\Rps}\shopiece{2}{5}{\Gs}\shopiece{-1}{9}{\Ps}\shopiece{-1}{8}{\Ns}\shopiece{1}{2}{\Lg}\shopiece{1}{3}{\Kg}\shopiece{2}{3}{\Pg}\shopiece{2}{4}{\Ng}}
\testwork{ベリアコフ・ダニラ}{Беляков Данила}{K77, +B96, P38, P55}{K11, +R65, +S61, +P88}{\shopiece{2}{5}{\Ks}\shopiece{1}{2}{\Bs}\shopiece{6}{6}{\Pps}\shopiece{6}{3}{\Pps}\shopiece{1}{9}{\Kg}\shopiece{9}{7}{\Bpg}\shopiece{8}{2}{\Lg}\shopiece{3}{5}{\Spg}}{\shopiece{1}{2}{\Rps}\shopiece{4}{5}{\Gs}\shopiece{3}{7}{\Bs}\shopiece{2}{4}{\Kg}\shopiece{2}{5}{\Pg}}{\shopiece{2}{4}{\Bps}\shopiece{2}{6}{\Bps}\shopiece{-1}{9}{\Gs}\shopiece{4}{3}{\Kg}\shopiece{5}{2}{\Pg}\shopiece{3}{2}{\Pg}}{\shopiece{3}{2}{\Rps}\shopiece{2}{5}{\Gs}\shopiece{-1}{9}{\Ps}\shopiece{-1}{8}{\Ns}\shopiece{1}{2}{\Lg}\shopiece{1}{3}{\Kg}\shopiece{2}{3}{\Pg}\shopiece{2}{4}{\Ng}}
\testwork{ドラガン・アリナ}{Драган Арина}{K54, B81, L67, +S84}{K27, B26, G69, L17}{\shopiece{7}{5}{\Ks}\shopiece{1}{5}{\Rps}\shopiece{8}{5}{\Pps}\shopiece{9}{9}{\Bps}\shopiece{8}{8}{\Kg}\shopiece{4}{8}{\Rpg}\shopiece{2}{3}{\Ppg}\shopiece{8}{7}{\Npg}}{\shopiece{4}{4}{\Gs}\shopiece{3}{7}{\Bs}\shopiece{1}{2}{\Rs}\shopiece{2}{3}{\Pg}\shopiece{2}{4}{\Kg}}{\shopiece{2}{4}{\Bps}\shopiece{2}{6}{\Bps}\shopiece{-1}{9}{\Gs}\shopiece{4}{3}{\Kg}\shopiece{5}{2}{\Pg}\shopiece{3}{2}{\Pg}}{\shopiece{3}{2}{\Rps}\shopiece{2}{5}{\Gs}\shopiece{-1}{9}{\Ps}\shopiece{-1}{8}{\Ns}\shopiece{1}{2}{\Lg}\shopiece{1}{3}{\Kg}\shopiece{2}{3}{\Pg}\shopiece{2}{4}{\Ng}}
\testwork{セミョノヴ天山スキイ・イェゴル}{Семенов Тян-Шанский Егор}{K53, +R85, P29, +P98}{K87, +R55, +P74, +P88}{\shopiece{3}{5}{\Ks}\shopiece{5}{7}{\Bs}\shopiece{4}{9}{\Ps}\shopiece{4}{7}{\Gs}\shopiece{8}{1}{\Kg}\shopiece{4}{5}{\Bpg}\shopiece{8}{6}{\Sg}\shopiece{8}{3}{\Ppg}}{\shopiece{1}{2}{\Rps}\shopiece{4}{5}{\Gs}\shopiece{3}{7}{\Bs}\shopiece{2}{4}{\Kg}\shopiece{2}{5}{\Pg}}{\shopiece{2}{4}{\Bps}\shopiece{2}{6}{\Bps}\shopiece{-1}{9}{\Gs}\shopiece{4}{3}{\Kg}\shopiece{5}{4}{\Pg}}{\shopiece{3}{2}{\Rps}\shopiece{2}{5}{\Gs}\shopiece{2}{7}{\Ps}\shopiece{-1}{9}{\Ps}\shopiece{-1}{8}{\Ns}\shopiece{1}{2}{\Lg}\shopiece{1}{3}{\Kg}\shopiece{1}{5}{\Pg}\shopiece{2}{3}{\Pg}\shopiece{1}{7}{\Rpg}}
\testwork{ナザロヴ・セルゲイ}{Назаров Сергей}{K81, B93, P47, +P43}{K67, +B37, +P41, G38}{\shopiece{1}{1}{\Ks}\shopiece{6}{3}{\Bps}\shopiece{5}{7}{\Ss}\shopiece{4}{5}{\Pps}\shopiece{9}{8}{\Kg}\shopiece{7}{4}{\Rg}\shopiece{3}{5}{\Gg}\shopiece{6}{7}{\Bg}}{\shopiece{1}{2}{\Rps}\shopiece{4}{5}{\Gs}\shopiece{3}{7}{\Bs}\shopiece{2}{4}{\Kg}\shopiece{2}{5}{\Pg}}{\shopiece{2}{4}{\Bps}\shopiece{2}{6}{\Bps}\shopiece{-1}{9}{\Gs}\shopiece{4}{3}{\Kg}\shopiece{3}{2}{\Ng}}{\shopiece{3}{2}{\Rps}\shopiece{2}{5}{\Gs}\shopiece{2}{7}{\Ps}\shopiece{-1}{9}{\Ps}\shopiece{-1}{8}{\Ns}\shopiece{1}{2}{\Lg}\shopiece{1}{3}{\Kg}\shopiece{1}{5}{\Pg}\shopiece{2}{3}{\Pg}\shopiece{1}{7}{\Rpg}}
\testwork{ドブレンコ・デニス}{Добренко Денис}{K25, +R98, G77, +P95}{K11, B67, P41, +S18}{\shopiece{4}{8}{\Ks}\shopiece{6}{7}{\Rps}\shopiece{9}{5}{\Ps}\shopiece{1}{5}{\Ns}\shopiece{6}{2}{\Kg}\shopiece{5}{7}{\Bpg}\shopiece{5}{3}{\Pg}\shopiece{8}{7}{\Pg}}{\shopiece{1}{2}{\Rps}\shopiece{4}{5}{\Gs}\shopiece{3}{7}{\Bs}\shopiece{2}{4}{\Kg}\shopiece{2}{5}{\Pg}}{\shopiece{2}{4}{\Bps}\shopiece{2}{6}{\Bps}\shopiece{-1}{9}{\Gs}\shopiece{4}{3}{\Kg}\shopiece{5}{4}{\Pg}}{\shopiece{3}{2}{\Rps}\shopiece{1}{5}{\Gs}\shopiece{-1}{9}{\Ns}\shopiece{1}{1}{\Lg}\shopiece{1}{3}{\Kg}\shopiece{2}{2}{\Lg}\shopiece{2}{3}{\Pg}\shopiece{4}{6}{\Bpg}}
\testwork{ツラヴィン・アレクサンデル}{Травин Александр}{K45, B21, G86, +P56}{K81, +R75, +N44, +L22}{\shopiece{9}{6}{\Ks}\shopiece{4}{6}{\Bs}\shopiece{4}{5}{\Pps}\shopiece{8}{5}{\Gs}\shopiece{2}{1}{\Kg}\shopiece{4}{4}{\Bg}\shopiece{3}{8}{\Sg}\shopiece{5}{1}{\Ppg}}{\shopiece{1}{2}{\Rps}\shopiece{4}{5}{\Gs}\shopiece{3}{7}{\Bs}\shopiece{2}{4}{\Kg}\shopiece{2}{5}{\Pg}}{\shopiece{2}{4}{\Bps}\shopiece{2}{6}{\Bps}\shopiece{-1}{9}{\Gs}\shopiece{4}{3}{\Kg}\shopiece{5}{4}{\Pg}}{\shopiece{3}{2}{\Rps}\shopiece{1}{5}{\Gs}\shopiece{-1}{9}{\Ns}\shopiece{1}{1}{\Lg}\shopiece{1}{3}{\Kg}\shopiece{2}{2}{\Lg}\shopiece{2}{3}{\Pg}\shopiece{4}{6}{\Bpg}}


%\testwork{アンヅリアノヴ・イリア}{Андрианов Илья}{\shopiece{7}{7}{\Ss}\shopiece{1}{1}{\Ks}\shopiece{7}{2}{\Ns}}{}
%\testwork{ベクレネヴァ・ダリア}{Бекренева Дарья}{}{}
%\testwork{ベリアコフ・ダニラ}{Беляков Данила}{}{}
%\testwork{ボルゾフ・アルテミイ}{Борзов Артемий}{}{}
%\testwork{ヴェショルコヴァ・ヴァルヴァラ}{Веселкова Варвара}{}{}
%\testwork{ヴァシェンコヴ・アレクセイ}{Ващенков Алексей}{}{}
%\testwork{ダニロヴァ・リュボヴィ}{Данилова Любовь}{}{}
%\testwork{ドラガン・アリナ}{Драган Арина}{}{}
%\testwork{ドブレンコ・デニス}{Добренко Денис}{}{}
%\testwork{ドニャマリエヴ・ルスラヌ}{Дунямалиев Руслан}{}{}
%\testwork{エヴドキモヴァ・アリョナ}{Eвдокимова Алёна}{}{}
%\testwork{イエメリアヌチク・ウラヂミル}{Емельянчик Владимир}{}{}
%\testwork{イワノヴ・アントヌ・アヌトノヴィチ}{Иванов Антон Антонович}{}{}
%\testwork{イワノヴ・アントヌ・オレゴヴィチ}{Иванов Антон Олегович}{}{}
%\testwork{カラコゾヴ・パヴェル}{Каракозов Павел}{}{}
%\testwork{ナザロヴ・セルゲイ}{Назаров Сергей}{}{}
%\testwork{ネマトヴァ・リナタ}{Ниматова Рената}{}{}
%\testwork{パルホメンコ・ゲオルギイ}{Пархоменко Георгий}{}{}
%\testwork{ルビナ・オレシャ}{Рубина Олеся}{}{}
%\testwork{セレブリャコヴ・イエヴゲニイ}{Серебряков Евгений}{}{}
%\testwork{セミョノヴ天山スキイ・イェゴル}{Семенов Тянь-Шанский Егор}{}{}
%\testwork{スホムリン・フィオドル}{Сухомлин Фёдор}{}{}
%\testwork{ツラヴィン・アレクサンデル}{Травин Александр}{}{}
%\testwork{トルカチョヴ・キリル}{Толкачёв Кирилл}{}{}
%\testwork{チェクマリョヴ・ヂミツリイ}{Чекмарёв Дмитрий}{}{}
%\testwork{チェミナワ・ソフィア}{Чеминава София}{}{}
%\testwork{チストヴ・オレグ}{Чистов Олег}{}{}
%\testwork{シェロヴ・ナザル}{Шеров Назар}{}{}

\end{CJK}



\end{document}
