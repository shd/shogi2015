\documentclass[12pt]{scrartcl}
\usepackage[russian]{babel}
\usepackage{CJKutf8}
\usepackage[usebaselinestretch]{CJKvert}
\usepackage[utf8]{inputenc}
\usepackage[width=18cm, height=28cm, top=0.5cm,bottom=0.5cm]{geometry}
\usepackage{rotating}
\usepackage[T2A]{fontenc}
\usepackage{sho}
\usepackage{setspace}
\usepackage{color}
\usepackage{epstopdf}
\usepackage{graphicx}

\definecolor{light-gray}{gray}{0.5}
\definecolor{name}{gray}{0.5}

\pagenumbering{gobble}% Remove page numbers (and reset to 1)

\newcommand\testwork[8]{
%\hspace{1cm}
\vspace{1cm}
{\Large 五級将棋クラブ}
\hfill\vspace{-6mm}
{\bfseries\large {#1}}
{\begin{center}\Huge 学期末試験\end{center}}
\vspace{-2mm}
{\begin{center}{\large 平成二十七年十二月二十四日}\\
\end{center}}
%\vspace{-2mm}
\vspace{5mm}


\begin{turn}{90}\begin{minipage}{20cm}
\begin{center}
{\setstretch{1.2}
\vspace{-1cm}
%\large Кружок сёги ЛНМО\\
{\bfseries\large\scshape Контрольная работа за вторую четверть}\\
{24 декабря 2015 г.}\\
{\itshape\tiny #2}}
\end{center}

\vspace{0.5cm}

\setstretch{1}

\begin{tabular}{lp{2cm}lp{2cm}}
\begin{minipage}[t]{8cm}
{\bfseries Задание 1. Фора}\vspace{0.5cm}

Укажите, с какой форой должна играться партия,
если первый игрок имеет #4, а второй --- #5.\vspace{2cm}

{\bfseries Задание 2. Теоретический вопрос}
\vspace{0.5cm}

Ответьте на теоретический вопрос:
\vspace{0.3cm}

{\it #3}\vspace{0.3cm}

При нехватке места ответ продолжайте на обороте листа. При необходимости используйте дополнительные листы.


\end{minipage}

& &


\begin{minipage}[t]{8cm}
%\begin{minipage}[t]{8cm}
{\bfseries Задание 3. Цумэ-сёги в 1 ход.}

%\vspace{5mm}
\begin{center}\includegraphics[width=5cm]{#6}\end{center}

%\end{minipage} & & %
%\begin{minipage}[t]{5cm}
\vspace{0.1cm}
{\it Ход чёрных: }
\vspace{0.5cm}

{\bfseries Задание 4. Цумэ-сёги в 3 хода.}
\begin{center}\includegraphics[width=5cm]{#7}\end{center}

%\end{minipage} & & %
\vspace{0.1cm}
{\it Ход чёрных/ответ белых/ход чёрных: }
\vspace{0.5cm}

\end{minipage} 

\end{tabular}

\vspace{0.5cm}

\begin{center}{\scshape Кружок сёги ЛНМО, 5 класс}\end{center}

\end{minipage}


\end{turn}


\vspace{1cm}

{\Huge \begin{center}\color{light-gray}数学連続教育研究所\end{center}}
\clearpage
\vspace{-1cm}
{\large 学期末試験 \hfill 第二頁}\\
\begin{turn}{90}\begin{minipage}{27.5cm}

\begin{center}
{\setstretch{1.2}
%\vspace{cm}
%\large Кружок сёги ЛНМО\\
{\large\scshape Контрольная работа за вторую четверть}\\
%{29 октября 2015 г.}\\
%{\itshape\tiny #2}
}
\end{center}

\setstretch{1.1}

\vspace{2cm}

\begin{tabular}{lp{0.2cm}lp{0.2cm}l}
\begin{minipage}[t]{8cm}{\bfseries Задание 5 (дополнительное)}

%\vspace{5mm}
\begin{center}\shodiag{1}{#8}\end{center}



Решите цумэ-сёги (мат в 9 ходов). Первый дан должен решать данную задачу за 10 минут.
\end{minipage} 

& &
Место для ответа на теоретический вопрос:

\end{tabular}

\vspace{3cm}
\begin{center}{\scshape Страница 2}\end{center}

\end{minipage}
\end{turn}

\clearpage
}

\begin{document}

\begin{CJK}{UTF8}{min}


\CJKvert

\testwork{エヴドキモヴァ・アリョナ}{Eвдокимова Алёна}{Перечислите правила цумэ-сёги}{4 дан}{2 кю}{q2-testwork/tsume1/2}{q2-testwork/tsume3/4}{\shopiece{5}{1}{\Rps}\shopiece{2}{5}{\Ns}\shopiece{-1}{9}{\Ns}\shopiece{-1}{8}{\Ss}\shopiece{-1}{7}{\Ss}\shopiece{-1}{6}{\Ss}\shopiece{1}{1}{\Lg}\shopiece{3}{2}{\Kg}\shopiece{5}{3}{\Bpg}\shopiece{4}{3}{\Pg}\shopiece{3}{3}{\Pg}\shopiece{2}{3}{\Pg}\shopiece{2}{4}{\Bg}}
\testwork{カラコゾヴ・パヴェル}{Каракозов Павел}{Перечислите правила цумэ-сёги}{4 дан}{2 дан}{q2-testwork/tsume1/18}{q2-testwork/tsume3/11}{\shopiece{5}{1}{\Rps}\shopiece{2}{5}{\Ns}\shopiece{-1}{9}{\Ns}\shopiece{-1}{8}{\Ss}\shopiece{-1}{7}{\Ss}\shopiece{-1}{6}{\Ss}\shopiece{1}{1}{\Lg}\shopiece{3}{2}{\Kg}\shopiece{5}{3}{\Bpg}\shopiece{4}{3}{\Pg}\shopiece{3}{3}{\Pg}\shopiece{2}{3}{\Pg}\shopiece{2}{4}{\Bg}}
\testwork{スホムリン・フィオドル}{Сухомлин Фёдор}{Дебют Шикенбиша --- последовательность ходов за чёрных, итоговая диаграмма после девятого хода}{1 кю}{3 дан}{q2-testwork/tsume1/8}{q2-testwork/tsume3/13}{\shopiece{5}{1}{\Rps}\shopiece{2}{5}{\Ns}\shopiece{-1}{9}{\Ns}\shopiece{-1}{8}{\Ss}\shopiece{-1}{7}{\Ss}\shopiece{-1}{6}{\Ss}\shopiece{1}{1}{\Lg}\shopiece{3}{2}{\Kg}\shopiece{5}{3}{\Bpg}\shopiece{4}{3}{\Pg}\shopiece{3}{3}{\Pg}\shopiece{2}{3}{\Pg}\shopiece{2}{4}{\Bg}}
\testwork{チストヴ・オレグ}{Чистов Олег}{Дебют Шикенбиша --- последовательность ходов за белых, итоговая диаграмма после девятого хода}{4 кю}{3 дан}{q2-testwork/tsume1/4}{q2-testwork/tsume3/19}{\shopiece{5}{1}{\Rps}\shopiece{2}{5}{\Ns}\shopiece{-1}{9}{\Ns}\shopiece{-1}{8}{\Ss}\shopiece{-1}{7}{\Ss}\shopiece{-1}{6}{\Ss}\shopiece{1}{1}{\Lg}\shopiece{3}{2}{\Kg}\shopiece{5}{3}{\Bpg}\shopiece{4}{3}{\Pg}\shopiece{3}{3}{\Pg}\shopiece{2}{3}{\Pg}\shopiece{2}{4}{\Bg}}
\testwork{ベクレネヴァ・ダリア}{Бекренева Дарья}{Дебют Санкенбиша (схема Исиды) --- последовательность ходов за чёрных, итоговая диаграмма}{5 кю}{3 дан}{q2-testwork/tsume1/14}{q2-testwork/tsume3/3}{\shopiece{5}{1}{\Rps}\shopiece{2}{5}{\Ns}\shopiece{-1}{9}{\Ns}\shopiece{-1}{8}{\Ss}\shopiece{-1}{7}{\Ss}\shopiece{-1}{6}{\Ss}\shopiece{1}{1}{\Lg}\shopiece{3}{2}{\Kg}\shopiece{5}{3}{\Bpg}\shopiece{4}{3}{\Pg}\shopiece{3}{3}{\Pg}\shopiece{2}{3}{\Pg}\shopiece{2}{4}{\Bg}}
\testwork{ボルゾフ・アルテミイ}{Борзов Артемий}{Дебют Йокофу --- последовательность ходов, вариант быстрой атаки, итоговая диаграмма}{5 дан}{3 дан}{q2-testwork/tsume1/6}{q2-testwork/tsume3/17}{\shopiece{5}{1}{\Rps}\shopiece{2}{5}{\Ns}\shopiece{-1}{9}{\Ns}\shopiece{-1}{8}{\Ss}\shopiece{-1}{7}{\Ss}\shopiece{-1}{6}{\Ss}\shopiece{1}{1}{\Lg}\shopiece{3}{2}{\Kg}\shopiece{5}{3}{\Bpg}\shopiece{4}{3}{\Pg}\shopiece{3}{3}{\Pg}\shopiece{2}{3}{\Pg}\shopiece{2}{4}{\Bg}}
\testwork{ルビナ・オレシャ}{Рубина Олеся}{Перечислите правила цумэ-сёги}{3 кю}{1 кю}{q2-testwork/tsume1/6}{q2-testwork/tsume3/10}{\shopiece{5}{1}{\Rps}\shopiece{2}{5}{\Ns}\shopiece{-1}{9}{\Ns}\shopiece{-1}{8}{\Ss}\shopiece{-1}{7}{\Ss}\shopiece{-1}{6}{\Ss}\shopiece{1}{1}{\Lg}\shopiece{3}{2}{\Kg}\shopiece{5}{3}{\Bpg}\shopiece{4}{3}{\Pg}\shopiece{3}{3}{\Pg}\shopiece{2}{3}{\Pg}\shopiece{2}{4}{\Bg}}
\testwork{シェロヴ・ナザル}{Шеров Назар}{Дебют Санкенбиша (схема Исиды) --- последовательность ходов за чёрных, итоговая диаграмма}{1 дан}{2 кю}{q2-testwork/tsume1/20}{q2-testwork/tsume3/18}{\shopiece{5}{1}{\Rps}\shopiece{2}{5}{\Ns}\shopiece{-1}{9}{\Ns}\shopiece{-1}{8}{\Ss}\shopiece{-1}{7}{\Ss}\shopiece{-1}{6}{\Ss}\shopiece{1}{1}{\Lg}\shopiece{3}{2}{\Kg}\shopiece{5}{3}{\Bpg}\shopiece{4}{3}{\Pg}\shopiece{3}{3}{\Pg}\shopiece{2}{3}{\Pg}\shopiece{2}{4}{\Bg}}
\testwork{ネマトヴァ・リナタ}{Нематова Рената}{Дебют Санкенбиша (схема Исиды) --- последовательность ходов за чёрных, итоговая диаграмма}{1 кю}{5 дан}{q2-testwork/tsume1/20}{q2-testwork/tsume3/13}{\shopiece{5}{1}{\Rps}\shopiece{2}{5}{\Ns}\shopiece{-1}{9}{\Ns}\shopiece{-1}{8}{\Ss}\shopiece{-1}{7}{\Ss}\shopiece{-1}{6}{\Ss}\shopiece{1}{1}{\Lg}\shopiece{3}{2}{\Kg}\shopiece{5}{3}{\Bpg}\shopiece{4}{3}{\Pg}\shopiece{3}{3}{\Pg}\shopiece{2}{3}{\Pg}\shopiece{2}{4}{\Bg}}
\testwork{イエメリアヌチク・ウラヂミル}{Емельянчик Владимир}{Поясните пословицу <<Пешка за золотом крепче скалы>>, приведите поясняющий пример}{4 кю}{5 дан}{q2-testwork/tsume1/15}{q2-testwork/tsume3/9}{\shopiece{5}{1}{\Rps}\shopiece{2}{5}{\Ns}\shopiece{-1}{9}{\Ns}\shopiece{-1}{8}{\Ss}\shopiece{-1}{7}{\Ss}\shopiece{-1}{6}{\Ss}\shopiece{1}{1}{\Lg}\shopiece{3}{2}{\Kg}\shopiece{5}{3}{\Bpg}\shopiece{4}{3}{\Pg}\shopiece{3}{3}{\Pg}\shopiece{2}{3}{\Pg}\shopiece{2}{4}{\Bg}}
\testwork{トルカチョヴ・キリル}{Толкачёв Кирилл}{Поясните пословицу <<Высоко прыгающий конь --- добыча для пешки>>, приведите поясняющий пример}{2 дан}{5 дан}{q2-testwork/tsume1/19}{q2-testwork/tsume3/9}{\shopiece{5}{1}{\Rps}\shopiece{2}{5}{\Ns}\shopiece{-1}{9}{\Ns}\shopiece{-1}{8}{\Ss}\shopiece{-1}{7}{\Ss}\shopiece{-1}{6}{\Ss}\shopiece{1}{1}{\Lg}\shopiece{3}{2}{\Kg}\shopiece{5}{3}{\Bpg}\shopiece{4}{3}{\Pg}\shopiece{3}{3}{\Pg}\shopiece{2}{3}{\Pg}\shopiece{2}{4}{\Bg}}
\testwork{アンヅリアノヴ・イリア}{Андрианов Илья}{Дебют Шикенбиша --- последовательность ходов за чёрных, итоговая диаграмма после девятого хода}{2 дан}{3 кю}{q2-testwork/tsume1/20}{q2-testwork/tsume3/1}{\shopiece{5}{1}{\Rps}\shopiece{2}{5}{\Ns}\shopiece{-1}{9}{\Ns}\shopiece{-1}{8}{\Ss}\shopiece{-1}{7}{\Ss}\shopiece{-1}{6}{\Ss}\shopiece{1}{1}{\Lg}\shopiece{3}{2}{\Kg}\shopiece{5}{3}{\Bpg}\shopiece{4}{3}{\Pg}\shopiece{3}{3}{\Pg}\shopiece{2}{3}{\Pg}\shopiece{2}{4}{\Bg}}
\testwork{ダニロヴァ・リュボヴィ}{Данилова Любовь}{Недопустимые ходы пешкой и конём}{4 дан}{1 дан}{q2-testwork/tsume1/11}{q2-testwork/tsume3/14}{\shopiece{5}{1}{\Rps}\shopiece{2}{5}{\Ns}\shopiece{-1}{9}{\Ns}\shopiece{-1}{8}{\Ss}\shopiece{-1}{7}{\Ss}\shopiece{-1}{6}{\Ss}\shopiece{1}{1}{\Lg}\shopiece{3}{2}{\Kg}\shopiece{5}{3}{\Bpg}\shopiece{4}{3}{\Pg}\shopiece{3}{3}{\Pg}\shopiece{2}{3}{\Pg}\shopiece{2}{4}{\Bg}}
\testwork{チェクマリョヴ・ヂミツリイ}{Чекмарёв Дмитрий}{Дебют Йокофу --- последовательность ходов, вариант медленной атаки, итоговая диаграмма, укажите опасный ошибочный ход (ловушку)}{1 дан}{4 дан}{q2-testwork/tsume1/6}{q2-testwork/tsume3/14}{\shopiece{5}{1}{\Rps}\shopiece{2}{5}{\Ns}\shopiece{-1}{9}{\Ns}\shopiece{-1}{8}{\Ss}\shopiece{-1}{7}{\Ss}\shopiece{-1}{6}{\Ss}\shopiece{1}{1}{\Lg}\shopiece{3}{2}{\Kg}\shopiece{5}{3}{\Bpg}\shopiece{4}{3}{\Pg}\shopiece{3}{3}{\Pg}\shopiece{2}{3}{\Pg}\shopiece{2}{4}{\Bg}}
\testwork{パルホメンコ・ゲオルギイ}{Пархоменко Георгий}{Крепость Мино --- ходы для построения, итоговый вид (диаграмма), укажите опасный ошибочный ход}{3 дан}{4 кю}{q2-testwork/tsume1/5}{q2-testwork/tsume3/2}{\shopiece{5}{1}{\Rps}\shopiece{2}{5}{\Ns}\shopiece{-1}{9}{\Ns}\shopiece{-1}{8}{\Ss}\shopiece{-1}{7}{\Ss}\shopiece{-1}{6}{\Ss}\shopiece{1}{1}{\Lg}\shopiece{3}{2}{\Kg}\shopiece{5}{3}{\Bpg}\shopiece{4}{3}{\Pg}\shopiece{3}{3}{\Pg}\shopiece{2}{3}{\Pg}\shopiece{2}{4}{\Bg}}
\testwork{ヴァシェンコヴ・アレクセイ}{Ващенков Алексей}{Недопустимые ходы пешкой и конём}{3 дан}{3 кю}{q2-testwork/tsume1/9}{q2-testwork/tsume3/11}{\shopiece{5}{1}{\Rps}\shopiece{2}{5}{\Ns}\shopiece{-1}{9}{\Ns}\shopiece{-1}{8}{\Ss}\shopiece{-1}{7}{\Ss}\shopiece{-1}{6}{\Ss}\shopiece{1}{1}{\Lg}\shopiece{3}{2}{\Kg}\shopiece{5}{3}{\Bpg}\shopiece{4}{3}{\Pg}\shopiece{3}{3}{\Pg}\shopiece{2}{3}{\Pg}\shopiece{2}{4}{\Bg}}
\testwork{チェミナワ・ソフィア}{Чеминава София}{Поясните пословицу <<Пешка за золотом крепче скалы>>, приведите поясняющий пример}{2 кю}{2 дан}{q2-testwork/tsume1/21}{q2-testwork/tsume3/1}{\shopiece{5}{1}{\Rps}\shopiece{2}{5}{\Ns}\shopiece{-1}{9}{\Ns}\shopiece{-1}{8}{\Ss}\shopiece{-1}{7}{\Ss}\shopiece{-1}{6}{\Ss}\shopiece{1}{1}{\Lg}\shopiece{3}{2}{\Kg}\shopiece{5}{3}{\Bpg}\shopiece{4}{3}{\Pg}\shopiece{3}{3}{\Pg}\shopiece{2}{3}{\Pg}\shopiece{2}{4}{\Bg}}
\testwork{ドニャマリエヴ・ルスラヌ}{Дунямалиев Руслан}{Дебют Йокофу --- последовательность ходов, вариант быстрой атаки, итоговая диаграмма}{2 дан}{1 кю}{q2-testwork/tsume1/11}{q2-testwork/tsume3/3}{\shopiece{5}{1}{\Rps}\shopiece{2}{5}{\Ns}\shopiece{-1}{9}{\Ns}\shopiece{-1}{8}{\Ss}\shopiece{-1}{7}{\Ss}\shopiece{-1}{6}{\Ss}\shopiece{1}{1}{\Lg}\shopiece{3}{2}{\Kg}\shopiece{5}{3}{\Bpg}\shopiece{4}{3}{\Pg}\shopiece{3}{3}{\Pg}\shopiece{2}{3}{\Pg}\shopiece{2}{4}{\Bg}}
\testwork{イワノヴ・アントヌ・オレゴヴィチ}{Иванов Антон Олегович}{Поясните пословицу <<Пешка за золотом крепче скалы>>, приведите поясняющий пример}{3 дан}{4 дан}{q2-testwork/tsume1/11}{q2-testwork/tsume3/21}{\shopiece{5}{1}{\Rps}\shopiece{2}{5}{\Ns}\shopiece{-1}{9}{\Ns}\shopiece{-1}{8}{\Ss}\shopiece{-1}{7}{\Ss}\shopiece{-1}{6}{\Ss}\shopiece{1}{1}{\Lg}\shopiece{3}{2}{\Kg}\shopiece{5}{3}{\Bpg}\shopiece{4}{3}{\Pg}\shopiece{3}{3}{\Pg}\shopiece{2}{3}{\Pg}\shopiece{2}{4}{\Bg}}
\testwork{セレブリャコヴ・イエヴゲニイ}{Серебряков Евгений}{Недопустимые ходы пешкой и конём}{1 кю}{1 кю}{q2-testwork/tsume1/4}{q2-testwork/tsume3/11}{\shopiece{5}{1}{\Rps}\shopiece{2}{5}{\Ns}\shopiece{-1}{9}{\Ns}\shopiece{-1}{8}{\Ss}\shopiece{-1}{7}{\Ss}\shopiece{-1}{6}{\Ss}\shopiece{1}{1}{\Lg}\shopiece{3}{2}{\Kg}\shopiece{5}{3}{\Bpg}\shopiece{4}{3}{\Pg}\shopiece{3}{3}{\Pg}\shopiece{2}{3}{\Pg}\shopiece{2}{4}{\Bg}}
\testwork{ヴェショルコヴァ・ヴァルヴァラ}{Веселкова Варвара}{Дебют Шикенбиша --- последовательность ходов за чёрных, итоговая диаграмма после девятого хода}{4 дан}{5 кю}{q2-testwork/tsume1/18}{q2-testwork/tsume3/17}{\shopiece{5}{1}{\Rps}\shopiece{2}{5}{\Ns}\shopiece{-1}{9}{\Ns}\shopiece{-1}{8}{\Ss}\shopiece{-1}{7}{\Ss}\shopiece{-1}{6}{\Ss}\shopiece{1}{1}{\Lg}\shopiece{3}{2}{\Kg}\shopiece{5}{3}{\Bpg}\shopiece{4}{3}{\Pg}\shopiece{3}{3}{\Pg}\shopiece{2}{3}{\Pg}\shopiece{2}{4}{\Bg}}
\testwork{イワノヴ・アントヌ・アヌトノヴィチ}{Иванов Антон Антонович}{Дебют Санкенбиша (схема Исиды) --- последовательность ходов за чёрных, итоговая диаграмма}{4 кю}{3 кю}{q2-testwork/tsume1/3}{q2-testwork/tsume3/20}{\shopiece{5}{1}{\Rps}\shopiece{2}{5}{\Ns}\shopiece{-1}{9}{\Ns}\shopiece{-1}{8}{\Ss}\shopiece{-1}{7}{\Ss}\shopiece{-1}{6}{\Ss}\shopiece{1}{1}{\Lg}\shopiece{3}{2}{\Kg}\shopiece{5}{3}{\Bpg}\shopiece{4}{3}{\Pg}\shopiece{3}{3}{\Pg}\shopiece{2}{3}{\Pg}\shopiece{2}{4}{\Bg}}
\testwork{ベリアコフ・ダニラ}{Беляков Данила}{Поясните пословицу <<Высоко прыгающий конь --- добыча для пешки>>, приведите поясняющий пример}{5 кю}{1 дан}{q2-testwork/tsume1/2}{q2-testwork/tsume3/1}{\shopiece{5}{1}{\Rps}\shopiece{2}{5}{\Ns}\shopiece{-1}{9}{\Ns}\shopiece{-1}{8}{\Ss}\shopiece{-1}{7}{\Ss}\shopiece{-1}{6}{\Ss}\shopiece{1}{1}{\Lg}\shopiece{3}{2}{\Kg}\shopiece{5}{3}{\Bpg}\shopiece{4}{3}{\Pg}\shopiece{3}{3}{\Pg}\shopiece{2}{3}{\Pg}\shopiece{2}{4}{\Bg}}
\testwork{ドラガン・アリナ}{Драган Арина}{Дебют Шикенбиша --- последовательность ходов за чёрных, итоговая диаграмма после девятого хода}{1 дан}{4 дан}{q2-testwork/tsume1/19}{q2-testwork/tsume3/16}{\shopiece{5}{1}{\Rps}\shopiece{2}{5}{\Ns}\shopiece{-1}{9}{\Ns}\shopiece{-1}{8}{\Ss}\shopiece{-1}{7}{\Ss}\shopiece{-1}{6}{\Ss}\shopiece{1}{1}{\Lg}\shopiece{3}{2}{\Kg}\shopiece{5}{3}{\Bpg}\shopiece{4}{3}{\Pg}\shopiece{3}{3}{\Pg}\shopiece{2}{3}{\Pg}\shopiece{2}{4}{\Bg}}
\testwork{セミョノヴ天山スキイ・イェゴル}{Семенов Тян-Шанский Егор}{Перечислите правила цумэ-сёги}{5 кю}{5 кю}{q2-testwork/tsume1/15}{q2-testwork/tsume3/21}{\shopiece{5}{1}{\Rps}\shopiece{2}{5}{\Ns}\shopiece{-1}{9}{\Ns}\shopiece{-1}{8}{\Ss}\shopiece{-1}{7}{\Ss}\shopiece{-1}{6}{\Ss}\shopiece{1}{1}{\Lg}\shopiece{3}{2}{\Kg}\shopiece{5}{3}{\Bpg}\shopiece{4}{3}{\Pg}\shopiece{3}{3}{\Pg}\shopiece{2}{3}{\Pg}\shopiece{2}{4}{\Bg}}
\testwork{ナザロヴ・セルゲイ}{Назаров Сергей}{Дебют Шикенбиша --- последовательность ходов за белых, итоговая диаграмма после девятого хода}{3 дан}{3 кю}{q2-testwork/tsume1/6}{q2-testwork/tsume3/15}{\shopiece{5}{1}{\Rps}\shopiece{2}{5}{\Ns}\shopiece{-1}{9}{\Ns}\shopiece{-1}{8}{\Ss}\shopiece{-1}{7}{\Ss}\shopiece{-1}{6}{\Ss}\shopiece{1}{1}{\Lg}\shopiece{3}{2}{\Kg}\shopiece{5}{3}{\Bpg}\shopiece{4}{3}{\Pg}\shopiece{3}{3}{\Pg}\shopiece{2}{3}{\Pg}\shopiece{2}{4}{\Bg}}
\testwork{ドブレンコ・デニス}{Добренко Денис}{Поясните пословицу <<Пешка за золотом крепче скалы>>, приведите поясняющий пример}{5 кю}{5 кю}{q2-testwork/tsume1/1}{q2-testwork/tsume3/9}{\shopiece{5}{1}{\Rps}\shopiece{2}{5}{\Ns}\shopiece{-1}{9}{\Ns}\shopiece{-1}{8}{\Ss}\shopiece{-1}{7}{\Ss}\shopiece{-1}{6}{\Ss}\shopiece{1}{1}{\Lg}\shopiece{3}{2}{\Kg}\shopiece{5}{3}{\Bpg}\shopiece{4}{3}{\Pg}\shopiece{3}{3}{\Pg}\shopiece{2}{3}{\Pg}\shopiece{2}{4}{\Bg}}
\testwork{ツラヴィン・アレクサンデル}{Травин Александр}{Дебют Йокофу --- последовательность ходов, вариант медленной атаки, итоговая диаграмма, укажите опасный ошибочный ход (ловушку)}{2 дан}{1 дан}{q2-testwork/tsume1/18}{q2-testwork/tsume3/1}{\shopiece{5}{1}{\Rps}\shopiece{2}{5}{\Ns}\shopiece{-1}{9}{\Ns}\shopiece{-1}{8}{\Ss}\shopiece{-1}{7}{\Ss}\shopiece{-1}{6}{\Ss}\shopiece{1}{1}{\Lg}\shopiece{3}{2}{\Kg}\shopiece{5}{3}{\Bpg}\shopiece{4}{3}{\Pg}\shopiece{3}{3}{\Pg}\shopiece{2}{3}{\Pg}\shopiece{2}{4}{\Bg}}


%\testwork{アンヅリアノヴ・イリア}{Андрианов Илья}{\shopiece{7}{7}{\Ss}\shopiece{1}{1}{\Ks}\shopiece{7}{2}{\Ns}}{}
%\testwork{ベクレネヴァ・ダリア}{Бекренева Дарья}{}{}
%\testwork{ベリアコフ・ダニラ}{Беляков Данила}{}{}
%\testwork{ボルゾフ・アルテミイ}{Борзов Артемий}{}{}
%\testwork{ヴェショルコヴァ・ヴァルヴァラ}{Веселкова Варвара}{}{}
%\testwork{ヴァシェンコヴ・アレクセイ}{Ващенков Алексей}{}{}
%\testwork{ダニロヴァ・リュボヴィ}{Данилова Любовь}{}{}
%\testwork{ドラガン・アリナ}{Драган Арина}{}{}
%\testwork{ドブレンコ・デニス}{Добренко Денис}{}{}
%\testwork{ドニャマリエヴ・ルスラヌ}{Дунямалиев Руслан}{}{}
%\testwork{エヴドキモヴァ・アリョナ}{Eвдокимова Алёна}{}{}
%\testwork{イエメリアヌチク・ウラヂミル}{Емельянчик Владимир}{}{}
%\testwork{イワノヴ・アントヌ・アヌトノヴィチ}{Иванов Антон Антонович}{}{}
%\testwork{イワノヴ・アントヌ・オレゴヴィチ}{Иванов Антон Олегович}{}{}
%\testwork{カラコゾヴ・パヴェル}{Каракозов Павел}{}{}
%\testwork{ナザロヴ・セルゲイ}{Назаров Сергей}{}{}
%\testwork{ネマトヴァ・リナタ}{Ниматова Рената}{}{}
%\testwork{パルホメンコ・ゲオルギイ}{Пархоменко Георгий}{}{}
%\testwork{ルビナ・オレシャ}{Рубина Олеся}{}{}
%\testwork{セレブリャコヴ・イエヴゲニイ}{Серебряков Евгений}{}{}
%\testwork{セミョノヴ天山スキイ・イェゴル}{Семенов Тянь-Шанский Егор}{}{}
%\testwork{スホムリン・フィオドル}{Сухомлин Фёдор}{}{}
%\testwork{ツラヴィン・アレクサンデル}{Травин Александр}{}{}
%\testwork{トルカチョヴ・キリル}{Толкачёв Кирилл}{}{}
%\testwork{チェクマリョヴ・ヂミツリイ}{Чекмарёв Дмитрий}{}{}
%\testwork{チェミナワ・ソフィア}{Чеминава София}{}{}
%\testwork{チストヴ・オレグ}{Чистов Олег}{}{}
%\testwork{シェロヴ・ナザル}{Шеров Назар}{}{}

\end{CJK}



\end{document}
