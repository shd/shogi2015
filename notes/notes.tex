\documentclass[10pt]{scrartcl}
\usepackage[russian]{babel}
\usepackage{CJKutf8}
%\usepackage[usebaselinestretch]{CJKvert}
\usepackage[utf8]{inputenc}
\usepackage[width=18cm, height=27cm, top=1.5cm]{geometry}
\usepackage{rotating}
\usepackage[T2A]{fontenc}
\usepackage{sho}
\usepackage{setspace}

\pagenumbering{gobble}% Remove page numbers (and reset to 1)

\begin{document}

\begin{CJK}{UTF8}{min}
\begin{center}{\bfseries\Large\scshape Памятка: немного о японской письменности}\\
\end{center}

\setstretch{1.1}
\vspace{0.2cm}

%Сёги - это традиционная японская игра, поэтому в них множество обозначений взято из 
%японской письменности. Для того, чтобы овладеть данной игрой, нам потребуется немного
%её изучить.

\begin{center}{\bfseries Иероглифы, используемые для обозначения фигур}
\vspace{2mm}

\begin{tabular}{|l|l|l|l|}
\hline
Фигура & Обозначение & Перевернутая фигура & Обозначение\\\hline
Король белых & 王&&\\\hline
Король чёрных & 玉&&\\\hline
Ладья & 飛& Дракон & 竜\\\hline
Слон & 角& Лошадь (дракон-лошадь) & 馬\\\hline
Золото & 金&&\\\hline
Серебро & 銀& Перевёрнутое серебро & 全\\\hline
Конь & 桂& Перевёрнутый конь & 圭\\\hline
Стрелка & 香& Перевёрнутая стрелка & 杏\\\hline
Пешка & 歩& Токин & と\\\hline
\end{tabular}\end{center}

\vspace{3mm}

\begin{center}{\bfseries Ещё несколько полезных иероглифов}
\vspace{2mm}

\begin{tabular}{|l|l|}
\hline
Иероглиф(ы) & Значение\\\hline
将棋 & Сёги (игра генералов)\\\hline
宿題  & Домашняя работа\\\hline
試験 & Проверочная работа\\\hline
級 & Класс\\\hline
%第一問 (дай-ити мон) – задача №1                     詰将棋 – цумэ-сёги
一二三四五六七八九十 & Числа от 1 до 10\\\hline
第 & Номер\\\hline
%;  百 = 100,  千 = 1000,  万 = 1E4,  億 = 1E8,  半 = 0,5
平成二十七年九月十八日 & 18 сентября 2015 года (Период хэйсэй, 27 год, 9 месяц, 18 день)%	得点 (токутэн) оценка/баллы
\\\hline
\end{tabular}\end{center}

\vspace{3mm}

\begin{center}{\bfseries Катакана}\end{center}
\vspace{-2mm}

\setstretch{1}
Катакана --- японская слоговая азбука, использующаяся преимущественно для записи иностранных слов.
Практически каждый знак в ней обозначает несколько звуков. Можно сравнить это с русскими
буквами \emph{е}, \emph{ё}, \emph{ю}, \emph{я}, которые также в некоторых ситуациях произносятся как сочетание нескольких звуков.

Буквы катаканы здесь сведены в таблицу --- строки соответствуют начальной части слога 
(ноль, один или два звука), столбцы --- завершающей части (гласному звуку).

\setstretch{1.1}

\begin{center}
\noindent\begin{tabular}{l|p{1.69cm}p{1.69cm}p{1.69cm}p{1.69cm}p{1.69cm}p{1.69cm}p{1.69cm}p{1.69cm}}
 & а & и & у & э & о & я & ю & ё\\
\hline
       &  ア а & イ и & ウ у & エ э & オ о & ヤ я & ユ ю& ヨ ё\\
к      &  カ ка &      キ  ки       &    ク  ку       &     ケ  кэ    & コ  ко  & キャ кя  &  キュ кю & キョ  кё \\
г      &   ガ  га  &  ギ  ги   &        グ  гу    &        ゲ  гэ      &       ゴ  го & ギャ гя &	ギュ гю &ギョ гё\\
с    &    サса    &   シ  си/ши &  ス  су    &        セ сэ     &        ソ  со &シャ ся &	シュ сю & ショ сё\\
дз   &            ザ дза  &    ジ  дзи   &      ズ  дзу      &     ゼ  дзэ   &        ゾ  дзо & ジャ дзя & ジュ дзю & ジョ дзё\\
т  &                 タ та  &     チ  ти/чи  &   ツ  цу    &         テ  тэ     &       ト  то&チャ тя &チュ тю &チョ тё\\
д  &                ダ  да    &  ヂ ди     &      ヅ  дзу    &       デ  дэ       &     ド  до & ヂャ (дзя) &ヂュ (дзю) &ヂョ (дзё)\\
н &                 ナ  на   &   ニ  ни     &     ヌ  ну       &      ネ  нэ    &        ノ  но & ニャ ня &ニュ ню &ニョ нё \\
х/ф &                 ハ  ха   &    ヒ  хи    &     フ  фу     &        ヘ  хэ      &      ホ  хо&ヒャ хя &ヒュ хю &ヒョ хё\\
б  &                バ  ба  &     ビ  би   &       ブ  бу     &         ベ  бэ     &        ボ  бо&ビャ бя &ビュ бю &ビョ бё\\
п  &                パ  па    &   ピ  пи      &   プ  пу     &       ペ  пэ        &     ポ  по&ピャ пя &ピュ пю &ピョ пё\\
ф  &                  ファ  фа &  フィ  фи     &                 &       フェ  фэ    &    フォ фо&フャ фя &フュ фю &フョ фё\\
м &                  マ ма    &  ミ  ми     &     ム  му   &         メ  мэ      &      モ  мо&ミャ мя &ミュ мю &ミョ мё\\
р  &                ラ  ра     &  リ  ри     &      ル  ру    &         レ  рэ     &        ロ  ро&リャ ря &リュ рю &リョ рё\\
в &                 ワ ва    &     ヰ  ви     &     ウ  ву  &    ヱ  вэ      &       ヲ  во
\end{tabular}
\end{center}

\setstretch{1}

Упомянем еще два часто используемых символа, не вошедших в таблицу:  ン 
(обозначает звук \emph{н}) и  ッ (удвоение согласной). Например, слово カッタ читается как \emph{катта}.
Это далеко не всё, в катакане существует множество других символов, которые используются реже.

\vspace{1cm}
\begin{center}{\scshape Кружок сёги ЛНМО, 5 класс}\end{center}
\end{CJK}

\end{document}
