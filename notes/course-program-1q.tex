\documentclass[12pt]{article}
\usepackage[russian]{babel}
\usepackage{CJKutf8}
%\usepackage[usebaselinestretch]{CJKvert}
\usepackage[utf8]{inputenc}
\usepackage[left=1.5cm,right=1.5cm,top=0.5cm]{geometry}
%\usepackage{rotating}
%\usepackage[T2A]{fontenc}
%\usepackage{sho}
\usepackage{setspace}

\pagenumbering{gobble}% Remove page numbers (and reset to 1)

\begin{document}

\begin{CJK}{UTF8}{min}
\setstretch{1.05}
\begin{center}{\bfseries\Large{\scshape Кружок сёги пятого класса ЛНМО:}\\\medbreak о зачёте за первую четверть}\end{center}

\vspace{0.3cm}
\noindent Зачёт будет проходить устно и письменно 29 октября 2015 года. Ученикам будет предложено несколько
вопросов на темы, приведённые ниже. Могут быть заданы как прямые вопросы на определения 
(<<скажите, что такое нифу>>), так и вопросы, требующие небольшого мыслительного усилия 
(<<сколько дополнительных полей начинает бить слон после переворота>>, 
<<назовите позицию ладьи чёрных в дебюте шикенбиша>>, <<в партии записан ход чёрных P*54; сделайте его на доске>>).
Иероглифы требуется уметь читать и различать, уметь писать их необязательно.
\bigbreak
\noindent Итоговая оценка за зачёт складывается из трёх оценок:
\begin{itemize}
\setlength{\parskip}{-1ex}\relax
\item[---] за ответы на вопросы;
\item[---] за решение простой задачи цумэ-сёги;
\item[---] за конспект.
\end{itemize}

\bigbreak
\noindent Ответы должны даваться только на основании памяти и размышления.
Никакими источниками (литературой, памятками, записями, конспектами и т.п.) при подготовке
ответов пользоваться будет нельзя.

\bigbreak
\begin{center}\bfseries\large Список тем\end{center}

\begin{enumerate}
\setlength{\parskip}{-0.3ex}\relax
\item Названия и ходы фигур.
\item Иероглифы, обозначающие фигуры (王玉飛竜角馬金銀全桂圭香杏歩と).
\item Японские цифры (一二三四五六七八九十).
\item Правила взятия и сброса фигур. 
\item Зона переворота. Правила превращения фигур. Ситуации, когда переворот обязателен.
\item Цель игры. Шах, цумэ, мат.
\item Фуригома. Сентэ и готэ.
\item Правила вежливости во время партии (что нужно, а что нельзя говорить; когда ход
считается сделанным; где и как хранить фигуры <<в руке>>).
\item Нифу и кинтэ. Когда и как заявлять об этих ситуациях.
\item Европейские обозначения фигур. Запись партии. Нумерация ходов. Запись перемещения фигуры, взятия фигуры,
переворота фигуры, сброса фигуры. Три способа записи позиций фигур на доске. 
\item Задачи на мат. Правила цумэ-сёги.
\item Система разрядов сёги (кю и даны).
\item Правила игр с форой. Правила определения величины форы в зависимости от разрядов противников.
\item Три части партии. Задачи дебюта.
\item Крепость Мино. Задачи крепостей.
\item Классификация дебютов по положению ладьи. Дебют шикенбиша, первые девять ходов за белых и за чёрных.
\end{enumerate}

%\setstretch{1.1}
%\vspace{0.2cm}
%\vspace{1cm}
%\begin{center}{\scshape Кружок сёги }\end{center}
\end{CJK}

\end{document}
