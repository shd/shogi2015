\documentclass[12pt]{article}
\usepackage[russian]{babel}
\usepackage[utf8]{inputenc}
\usepackage[left=1.5cm,right=1.5cm,top=0.5cm]{geometry}
\usepackage{setspace}

\pagenumbering{gobble}% Remove page numbers (and reset to 1)

\begin{document}

\setstretch{1.05}
\begin{center}{{\scshape\large Кружок сёги пятого класса ЛНМО}\bfseries\\\medbreak \scshape\Large О зачёте за вторую четверть}\end{center}

\vspace{1cm}
\noindent Зачёт будет проходить письменно 24 декабря 2015 года. Ученикам будет предложено 
решить несколько несложных задачек, а также изложить на бумаге ответ на теоретический вопрос.
\bigbreak
\noindent Итоговая оценка за четверть складывается из трёх оценок:
\begin{itemize}
\setlength{\parskip}{-1ex}\relax
\item[---] средней за текущую работу в течение четверти;
\item[---] за конспект (конспект будет проверен во время зачёта);
\item[---] за зачёт.
\end{itemize}

\bigbreak
\noindent Ответы должны даваться только на основании памяти и размышления.
Никакими источниками (литературой, памятками, записями, конспектами и т.п.) при подготовке
ответов пользоваться будет нельзя.

\bigbreak
\noindent При подготовке, при возникновении спорных вопросов, рекомендуется пользоваться дополнительной 
литературой с сайта \texttt{shogi.ru}, в частности следующими книгами:
\begin{itemize}
\setlength{\parskip}{-1ex}\relax
\item[---] Урано Масахико. Пословицы cёги. --- \texttt{http://shogi.ru/lib/kakugen.rar}
\item[---] Тони Хоскинг. От дебюта к миттельшпилю --- \texttt{http://shogi.ru/lib/Hosking2.rar}
\end{itemize}

\bigbreak
\begin{center}\bfseries\large Список тем для теоретических вопросов\end{center}

\begin{enumerate}
\setlength{\parskip}{-0.3ex}\relax
\item Дебют Шикенбиша --- последовательность ходов за белых или за чёрных, итоговая диаграмма.
\item Дебют Санкенбиша (модель Исиды) --- последовательность ходов, итоговая диаграмма.
\item Дебют Йокофу (Йокофудори) --- последовательность ходов, варианты атаки, итоговая диаграмма.
\item Крепость Мино.
\item Правила цумэ-сёги.
\item Пословицы сёги --- пословицы <<Пешка за золотом крепче скалы>>, <<Сбрасывай стрелку на нижнюю горизонталь>>,
<<Высоко прыгающий конь --- добыча для пешки>>, с поясняющими примерами.
\item Недопустимые ходы.
\end{enumerate}

\bigbreak
\begin{center}\bfseries\large Список тем для задач\end{center}

\begin{enumerate}
\setlength{\parskip}{-0.3ex}\relax
\item Решить цумэ-сёги.
\item Определить фору.
\item Проверить корректность позиции или хода.
\end{enumerate}

\end{document}
